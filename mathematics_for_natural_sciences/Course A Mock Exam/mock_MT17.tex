
% Exam Template for UMTYMP and Math Department courses
%
% Using Philip Hirschhorn's exam.cls: http://www-math.mit.edu/~psh/#ExamCls
%
% run pdflatex on a finished exam at least three times to do the grading table on front page.
%
%%%%%%%%%%%%%%%%%%%%%%%%%%%%%%%%%%%%%%%%%%%%%%%%%%%%%%%%%%%%%%%%%%%%%%%%%%%%%%%%%%%%%%%%%%%%%%

% These lines can probably stay unchanged, although you can remove the last
% two packages if you're not making pictures with tikz.
\documentclass[11pt]{exam}
\RequirePackage{amssymb, amsfonts, amsmath, latexsym, verbatim, xspace, setspace}

% By default LaTeX uses large margins.  This doesn't work well on exams; problems
% end up in the "middle" of the page, reducing the amount of space for students
% to work on them.
\usepackage[margin=1in]{geometry}


% Here's where you edit the Class, Exam, Date, etc.
\newcommand{\tripos}{Natural Sciences Tripos Part 1A}
\newcommand{\term}{Lent Term 2018}
\newcommand{\exam}{Course A Mathematics Mock Exam}
\newcommand{\timelimit}{2 hours}
\newcommand{\supervisor}{Nora Martin}

\newcommand{\myvec}[1]{\ensuremath{\begin{pmatrix}#1\end{pmatrix}}}

% For an exam, single spacing is most appropriate
\singlespacing
% \onehalfspacing
% \doublespacing

% For an exam, we generally want to turn off paragraph indentation
\parindent 0ex

\begin{document} 

% These commands set up the running header on the top of the exam pages
\pagestyle{head}
\firstpageheader{}{}{}
\runningheader{\exam}{Page \thepage\ of \numpages}{\term}
\runningheadrule

\begin{flushright}
\begin{tabular}{p{2.8in} r l}
\textbf{\tripos} & \textbf{Name (Print):} & \makebox[2in]{\hrulefill}\\
\textbf{\term} &&\\
\textbf{\exam} &&\\
\textbf{Time Limit: \timelimit}&&\\
\textbf{Supervisor: \supervisor}&&\\
\end{tabular}\\
\end{flushright}
\rule[1ex]{\textwidth}{.1pt}
\hspace{20pt}

Please attempt all questions in Part A and 3 questions in Part B.\\

Good luck!\\

\begin{minipage}[t]{3.7in}
\vspace{0pt}

\end{minipage}
\hfill
\begin{minipage}[t]{2.3in}
\vspace{0pt}
%\cellwidth{3em}
\gradetablestretch{2}
\vqword{Problem}
\addpoints % required here by exam.cls, even though questions haven't started yet.	
\gradetable[v]%[pages]  % Use [pages] to have grading table by page instead of question

\end{minipage}
\newpage % End of cover page

%%%%%%%%%%%%%%%%%%%%%%%%%%%%%%%%%%%%%%%%%%%%%%%%%%%%%%%%%%%%%%%%%%%%%%%%%%%%%%%%%%%%%
%
% See http://www-math.mit.edu/~psh/#ExamCls for full documentation, but the questions
% below give an idea of how to write questions [with parts] and have the points
% tracked automatically on the cover page.
%
%
%%%%%%%%%%%%%%%%%%%%%%%%%%%%%%%%%%%%%%%%%%%%%%%%%%%%%%%%%%%%%%%%%%%%%%%%%%%%%%%%%%%%%
\section*{Section A}\label{sa}
\begin{questions}

\addpoints
\question[2] Given $\vec{OC} = (1,2,3)$ and $\vec{AC} = (2,−1,−5)$, find an expression for the cosine of the angle between vectors $\vec{OA}$ and $\vec{OC}$.

%Complex Numbers and log
\addpoints
\question[4]Evaluate $\ln(\frac{1}{\sqrt{2}} (1+i))$ and show the result on the complex plane.


%plotting
\addpoints
\question
\begin{parts}
\part[2] Calculate the second derivative with respect to x of the real function $y = \sin(\exp x)$.
\part[3] Sketch the function $y = \sin(\exp x)$ in the range $-\infty < x < \infty$.
\end{parts}

%claculus
\addpoints
\question[6] Find the zeros, the stationary points and the inflection points of $y = x^{3} - 3 x^{2} + 4$
and state whether the stationary points are maxima or minima. Indicate these points on a graph of the function.



%Taylor Series question
\addpoints
\question
\begin{parts}
\part[3] State Taylor's theorem for the expansion about $x =x_{0}$ of a function that is differentiable $n$ times and give an expression for the remainder term $R_{n}$ after $n$ terms.
\part[5] Give the first three terms of the Taylor Series of $f(x)=\exp(-2 x)$ around $x=0$. What is the $n^{th}$ term of this series?
\end{parts}



\section*{Section B}\label{sb}
%Geometry Question
\addpoints
\question Two planes are defined by the equations:\\
Plane A: $x+y-2z=1$\\

Plane B: $x=2$\\
\begin{parts}
\part[4] A third plane, plane C, is defined by two lines. These lines are:
\begin{equation}
\vec{x}=\myvec{1\\2\\3}+r \myvec{1\\1\\-4}
\end{equation}
\begin{equation}
\vec{x}=\myvec{5\\-4\\-3}+s \myvec{-2\\3\\3}
\end{equation}
Find a coordinate equation of plane C.
\part[4] Calculate the volume of the parallelepiped of side lengths $2$, $3$ and $5$, which exactly occupies one of the corners of the point of intersection of planes A, B and C.
\part[4] Find the distance of plane A from the origin. Also find the coordinates of the point at the foot of the perpendicular from the origin to this plane. 
\end{parts}

%Calculus Question
\addpoints
\question Integration:
\begin{parts}
\part[2] Evaluate from first principles, by considering elementary areas, the integral:
\begin{equation}
\int_{a}^{b}x\ dx
\end{equation}

\part[4] Evaluate the integral:
\begin{equation}
\int_{0}^{\infty} e^{-x}\  sin(2x)\  dx
\end{equation}
\part[6] Find a recurrence relation between $I_{n}$ and $I_{n-1}$  for 
\begin{equation}
I_{n}=\int_{0}^{\infty} x^{n} e^{-x}\ dx % problem set question
\end{equation}
and use it to calculate $I_{n}$ for the case $n = 5$.\\
\end{parts}

%Complex Number Question
\addpoints
\question Complex Numbers I \textit{(adapted from Tripos paper 1, 2002)}:
\begin{parts}
\part[5] Let $z=2\exp(i \phi)$ where $0<\phi<\pi/2$. Express $z^{*}$ and $-z$ in the form $r \exp(i \theta)$ where $r > 0$ and $0 < \theta < 2 \pi$. For $\phi=\pi / 4$, sketch the location of $z$, $z^{*}$ and $-z$ in the complex plane.
\part[3] If $z = 2 \exp(i \phi)$ where $0 < \phi < \pi$, calculate the real and imaginary parts of $w = (z - 2)/(z + 2)$.
\part[4] Write down formulae for $\cos \theta$ and $\sin \theta$ in terms of complex exponentials. Use
these to derive formulae for $\cos 2\theta$ and sin $\sin 2\theta$  in terms of $\cos \theta$  and $\sin \theta$ . 
\end{parts}

%Complex Number Question
\addpoints
\question Complex Numbers II \textit{(adapted from Tripos paper 1, 2008)}:
\begin{parts}
\part[4] Let $z = x+i\ y$ with $x$ and $y$ real. Find the real and imaginary parts of the following
in terms of $x$ and $y$: $z \sin (z)$
\part[3] Find all the roots of the equation $z^{4} - z^{2} -2 = 0$ and plot them in the Argand diagram.
\part[5] Let the complex numbers $z_{1},z_{2},z_{3}$ and $z_{4}$ represent the vertices of a plane quadrilateral ABCD in the complex plane. Show that ABCD is a parallelogram if $z_{1} - z_{2} + z_{3} - z_{4} = 0$.
\end{parts}



%Probability Question
\addpoints
\question Probabilities:
\begin{parts}
\part[5]   Find the probability that in a group of k people at least two have the same birthday (ignoring 29th February).
\part[7] The random variable $X$ can take any non-negative value and its \textbf{cumulative} density function is $P(0\leq X \leq x)=k\ (1-\exp(-x)) $, where $k$ is a constant. Find the probability \textbf{density} function for $X$, the value of $k$, and the mean and variance of $X$.
\end{parts}



\end{questions}
\end{document}
