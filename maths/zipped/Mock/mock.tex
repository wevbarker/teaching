%mock exam for NST 1A math
\documentclass[10pt]{iopart}
\usepackage{multicol}
\usepackage[left=3.0cm]{geometry}
\usepackage[utf8]{inputenc}
\usepackage{graphicx}
\usepackage{caption}
\usepackage{alltt}
\usepackage{url}
\usepackage{textcomp}
\usepackage{booktabs}
\usepackage{iopams}
\usepackage[T1]{fontenc}
\pagestyle{empty}
\usepackage{csquotes}
\usepackage{enumitem}
\usepackage{amssymb}
\begin{document}
\title{Mock NST 1A math paper: things done in Michaelmas\\ \vspace{1cm} \large 14:00-16:00, Wednesday 17\textsuperscript{th} January 2018}
\vspace{2cm}
\begin{itemize}
\item	Try \textbf{all} the problems in \textbf{Section A} and any \textbf{three} problems in \textbf{Section B}. 
\item	About $1/4$ of the marks are in \textbf{Section A}, $3/4$ in \textbf{Section B}. 
\item	There are no other guidelines for mark distributions.
\item	Don't bother with a calculator.
\item	Problems recommended for the B-course folks have a star, (*).
\end{itemize}
\maketitle
\section*{Section A}\label{sa}

Try \textbf{all} of these in about 1/2 hour.
\begin{enumerate}[label=A\arabic*)]
\item	Use the Cauchy-Schwarz inequality to show that
\begin{eqnarray}
\int_{-\infty}^{\infty}\left[\cosh(\theta)\left(1+\theta^2\right)^{3/4}\right]^{-1}\mathrm{d}\theta\leq 2.
\end{eqnarray}
\item	Three planes are defined by the equations
\begin{eqnarray}
x+y-2z=12,\\
2x-2y+z=1, \\
3x+y+z=7.
\end{eqnarray}  
Find the volume of the parallelepiped of side lengths $2$, $3$ and $5$, which exactly occupies one of the corners of their point of intersection.
\item	The region $\Gamma_\xi$ in the $x-y$ plane is bounded by the lines $y=f_\xi(x)$, $x=b_\xi$ and $y=a_\xi$, where $f_\xi(x)$ is a monotonically decreasing function of $x$ for all values of the parameter $\xi$ (in this problem the $\xi$-subscript indicates implicit dependence on $\xi$). By considering the $\xi$-dependence of the double integral of some function $g_\xi(x,y)$ over this region,
\begin{eqnarray}
I_\xi=\int_{\Gamma_\xi}g_\xi(x,y)\mathrm{d}x\mathrm{d}y,
\end{eqnarray}
show that,
\begin{eqnarray}
\int_{b_\xi}^{f^{-1}_\xi(a_\xi)}g_\xi(u,f_\xi(u))\frac{\partial}{\partial\xi}f_\xi(u)\mathrm{d}u=\int_{a_\xi}^{f_\xi(b_\xi)}g_\xi(f^{-1}_\xi(u),u)\frac{\partial}{\partial\xi}f^{-1}_\xi(u)\mathrm{d}u.
\end{eqnarray}
\item	Show that
\begin{eqnarray}
8e^{i\pi/4}\int_{0}^{\infty}\theta^5\left[e^{i\pi/4}\sin\left(i\sqrt[3]{\pi}\theta^4\right)+e^{-i\pi/4}\cos\left(i\sqrt[3]{\pi}\theta^4\right)\right]\mathrm{d}\theta=1.
\end{eqnarray}
\item	Show that the number of \textit{disconnected} loci of $\mathbf{x}$ simultaneously satisfying \textit{both} constraints,
\begin{eqnarray}
|\mathbf{x}\times\hat{\mathbf{n}}|=\chi|\mathbf{x}|,\\ |\mathbf{x}-\lambda\hat{\mathbf{n}}|=\psi,
\end{eqnarray}
is two when $|\lambda|<\psi/\chi$, one when $|\lambda|=\psi/\chi$ and zero otherwise, where $\chi$ and $\psi$ are constant scalars and $\hat{\mathbf{n}}$ is a constant unit vector\footnote{Note that a locus can be a region of any dimension from $0$ to $3$.}.
\item	Primed and unprimed coordinates for a three-dimensional Euclidean space are related by
\begin{eqnarray}
\eqalign{
x_0=\cosh(\psi)x'_0-\sinh(\psi)x'_1,\\
x_1=\cosh(\psi)x'_1-\sinh(\psi)x'_0,\\
x_2=x'_2,
}
\end{eqnarray}
where $\psi$ is a parameter. Find the normalized basis vectors $\{\mathbf{e}'_i\}$ in terms of the Cartesian basis vectors $\{\mathbf{e}_i\}$ and verify that orthogonality has only been lost between the $x'_0$ and $x'_1$ coordinates. Find the reciprocal basis set $\{\mathbf{E}'_i\}$ in terms of the $\{\mathbf{e}_i\}$.
\end{enumerate}
\item   Define the terms \emph{continuity} and \emph{differentiability} in relation to a function of a real variable. A function is given by
\begin{eqnarray}
		f(x) &= \exp\bigg(\frac{-1}{x}\bigg) & &x >0 \\
		f(x) &= 0 & &\mathrm{otherwise.}
\end{eqnarray}
Determine whether $f$ is continuous and differentiable at the origin, and give a sketch of the function's graph.
\item   Find the real and imaginary parts of the square roots of the complex number $1-\sqrt{3}i$.


\section*{Section B}\label{sb}
Try any \textbf{three} of these, about 1/2 hour each.
\begin{enumerate}[label=B\arabic*)]
\item	The path $P$ is made up from an infinite series of steps. The $n$th step involves travelling a distance $l_n=l_0\varepsilon^n/n$ in a straight line and then taking a left turn through an angle $\theta$, so that $\varepsilon$ and $\theta$ are dimensionless constants and $l_0$ has dimensions of length. 

\hfill\break
Sketch the first four steps of the path $\varepsilon=1/2$ and $\theta=\pi/4$, beginning at $n=1$.

\hfill\break
Show that the series
\begin{eqnarray}
S^{N}(\varepsilon)=\sum_{n=1}^{N}\frac{\varepsilon^n}{n}, \quad \varepsilon\in\mathbb{R}
\end{eqnarray}
is convergent for $|\varepsilon|<1$ but divergent for $|\varepsilon|>1$, and that convergence for $-1<\varepsilon<0$ is absolute.

\hfill\break
Show further, with reference to definite integrals of the form $\int_{x_1}^{x_2}\mathrm{d}x/x$, that $S^N(\varepsilon)$ is divergent at $\varepsilon=1$ and conditionally convergent at $\varepsilon=-1$.

\hfill\break
Hence and by means of differentiation also, show that the Taylor series for the real natural logarithm of $x\in\mathbb{R}$ and its range of validity are
\begin{eqnarray}\label{one}
\ln(x)=\sum_{n=1}^{\infty}\frac{(-1)^{n+1}(x-1)^n}{n}, \quad -1<x-1\leq 1.
\end{eqnarray}

\hfill\break
Given that in \eref{one} the series expansion (known as the Newton-Mercator series) is also valid for the particular natural logarithm of $x\in\mathbb{C}$ for which $\pi>\mathfrak{I}[\ln(x)]>-\pi$, with the range $|x-1|\leq1$ and $x\neq 2$, show that a traveller on $P$ with $\varepsilon=1$ and $\theta=\pi/4$ must travel infinitely far but will eventually find herself a finite distance,
\begin{eqnarray}
l=l_0\sqrt{\left[\frac{1}{2}\ln\left(2-\sqrt{2}\right)\right]^2+\left[\arctan\left(\frac{1}{1-\sqrt{2}}\right)\right]^2},
\end{eqnarray}
from her starting point.
\item   A continuous random variable $X$ has the probability distribution function
\begin{eqnarray}
		f(x) &= \frac{A}{\sqrt{1+k^{2}x^{2}}} & &0\leq x\leq k^{-1} \\
		f(x) &= 0 & &\mathrm{otherwise,}
\end{eqnarray}
with $k$ a positive constant and $A$ a normalisation factor. Find the expectation value of $X$ in terms of $k$.

\hfill\break
A random variable $Y$ can take the value $-1$ or any non-negative value, according to
\begin{eqnarray}
	\mathrm{P}\left(Y = -1\right)=m; \qquad \mathrm{P}\left(0\leq Y\leq y\right) = \frac{1-e^{-y}}{2},
\end{eqnarray}
where $m$ is a constant. Find:
\begin{enumerate}[label=(\roman*)]
\item the value of $m$;
\item the mean of $Y$;
\item the variance of $Y$;
\item the probability that $Y\geq 1$ given that $Y\geq 0$.
\end{enumerate}



\item
\item
\item
\end{enumerate}
\end{document}
